\specialsection{结\ \ \ \ \ \ \ \ 论}
% \section*{结\ \ \ \ \ \ \ \ 论}
% \addcontentsline{toc}{section*}{结\ \ \ \ \ \ \ \ 论}


	本文针对AZ31B镁合金板材和1060纯铝板材在辊面温度300℃条件下进行复合轧制,成功制备出了Al/Mg/Al层状复合板材。研究了在不同轧制压下率条件和不同退火温度条件下的显微组织性能与力学性能变化,主要结论如下:\par
	\begin{enumerate}
		\item 随着轧制变形量的增加,复合板材镁层、铝层厚度不断降低、第一道次下降幅度较大,随后变化幅度较低,这是由于加工硬化导致变形能力减弱。中心镁层组织随着塑性变形的累积,由原始母材的粗大等轴晶先被拉长成为细长晶粒后变成细小的等轴晶。
		
		\item 通过XRD衍射图谱和SEM线扫描图谱分析,当压下率较低时轧后板材未发现金属间化合物,但在后续扩散退火中析出第二相,而压下率为73$\%$和83$\%$时轧制候复合板材出现金属间化合物Mg17Al12和Al3Mg2。由于加工硬化机制和细晶强化机制共同作用,复合板材的综合力学性能得到提升。
		
		\item 退火处理可以消除残余应力和减少内部缺陷,退火温度升高促进了异种材料界面处镁铝原子相互扩散。 
				
	\end{enumerate}
	